\documentclass[10pt]{article}

\usepackage[utf8]{inputenc}
\usepackage[spanish]{babel}

\usepackage[hidelinks]{hyperref}

\usepackage[a4paper, lmargin=0.14\paperwidth, rmargin=0.14\paperwidth, tmargin=0.1111\paperheight, bmargin=0.1111\paperheight]{geometry}


\usepackage{graphicx}
\graphicspath{ {images/} }


\usepackage{fancyhdr} % Headers and footers
\pagestyle{fancy} % All pages have headers and footers
\fancyhead{} % Blank out the default header
\fancyfoot{} % Blank out the default footer
\fancyhead[C]{ \today \ $\bullet$ PyS $\bullet$ Consumo Energético de los Centros de Computación } % Custom header text
\fancyfoot[RO] {\thepage}



\begin{document}

%----------------------------------------------------------------------------------------
%	TITLE SECTION
%----------------------------------------------------------------------------------------
	\begin{titlepage}
      \centering
          %\includegraphics[width=0.15\textwidth]{example-image-1x1}\par\vspace{1cm}
          {\scshape\LARGE Universidad de Valladolid \par}
          \vspace{1cm}
          {\scshape\Large Profesión y Sociedad.\par}
          \vspace{1.5cm}
          {\huge\bfseries Consumo Energético de los Centros de Computación \par}
          \vspace{2cm}
          {\large
          \textsc{Amigo Alonso, Alberto}\\[2mm] % Your name
          \textsc{Delgado Álvarez, Sergio}\\[2mm] % Your name
          \textsc{García Prado, Sergio}\\[2mm] % Your name
          \textsc{Iglesias Cortijo, David}\\[2mm] % Your name

          \vspace{-5mm}
          }

          \vfill
		% Bottom of the page
		{\large \today\par}
	\end{titlepage}

%----------------------------------------------------------------------------------------
%	TABLE OF CONTENTS
%----------------------------------------------------------------------------------------

	\clearpage
	\tableofcontents

%----------------------------------------------------------------------------------------
%	TEXT
%----------------------------------------------------------------------------------------

	\clearpage
    \section{Introducción}
	\label{sec:introducion}

        \paragraph{}


    \section{Análisis sobre el consumo}
    \label{sec:analisis}
    	\paragraph{}
		El primer paso en este estudio será analizar cuantitativamente el consumo energético de un centro de computación

    \section{Impacto medioambiental}
	\label{sec:impacto}

        \paragraph{}
		Los centros de computación provocan una gran alteración del medio ambiente, ya que requieren mucha energía para poder funcionar y además desprenden grandes cantidades de calor. Por tanto el impacto medioambiental es grande. A continuación se tratarán las formas en las que los centros de computación afectan gravemente al medio ambiente.
        
        \subsection{Gasto energético}
        	\paragraph{}
        
        \subsection{Impacto hidráulico}
        	\paragraph{}
            
        \subsection{Emisiones de dióxido carbono}
        	\paragraph{}
            
        \subsection{Impacto sobre el suelo terrestre}
        	\paragraph{}
            
        \subsection{Fauna y flora del entorno}
        	\paragraph{}
            
        
            
            

    \section{Estrategias de optimización}
	\label{sec:estrategias}

    	\paragraph{}
		Como ya se ha visto en la sección \ref{sec:analisis}, los grandes centros de computación producen un elevado consumo energético, lo que repercute negativamente en la productividad de los mismos, y por lo tanto en los beneficios económicos. Además, tal y como vimos en la sección \ref{sec:impacto}, el entorno medioambiental en la zona donde estos se localizan puede verse afectado negativamente.


		\paragraph{}
		Debido a estos factores, las organizaciones encargadas de gestionar este tipo de centros, cada vez más, dedican un alto grado de esfuerzo para tratar de reducir su consumo energético. Para ello utilizan distintas estrategias. En este documento vamos a dividir estas técnicas según el objetivo al que van dirigidas:


		\subsection{Refrigeración}

			\paragraph{}


		\subsection{Tiempo en estado de Espera}

			\paragraph{}


		\subsection{Otras}

			\paragraph{}


    \section{Beneficios}
	\label{sec:beneficios}

    	\paragraph{}

    \section{Conclusiones}
	\label{sec:conclusiones}

    	\paragraph{}
%----------------------------------------------------------------------------------------
%	Bibliographic references
%----------------------------------------------------------------------------------------
	\clearpage
    \begin{thebibliography}{9}

        \bibitem{energy-star:guide}
  		Enegy Star: 12 Way save \url{https://www.energystar.gov/products/low_carbon_it_campaign/12_ways_save_energy_data_center}

        \bibitem{intel:guide}
        Intel: Reducing Data Center Energy Consumption \url{https://www.irif.fr/~yunes/divers/papers/green/CERN_r04.pdf}

        \bibitem{google:ia}
        Google: Blog \url{https://googleblog.blogspot.com.es/2014/05/better-data-centers-through-machine.html}

        \bibitem{google:deep-learning}
        Deepmind \url{https://deepmind.com/blog/deepmind-ai-reduces-google-data-centre-cooling-bill-40/}

        \bibitem{nrdc:guide}
        NRDC: Data Center Eficiency Assessment \url{https://www.nrdc.org/sites/default/files/data-center-efficiency-assessment-IP.pdf}

	\end{thebibliography}

\end{document}
