\documentclass[10pt]{article}

\usepackage[utf8]{inputenc}
\usepackage[spanish]{babel}

\usepackage[hidelinks]{hyperref}

\usepackage[a4paper, lmargin=0.14\paperwidth, rmargin=0.14\paperwidth, tmargin=0.1111\paperheight, bmargin=0.1111\paperheight]{geometry}


\usepackage{graphicx}
\graphicspath{ {images/} }


\usepackage{fancyhdr} % Headers and footers
\pagestyle{fancy} % All pages have headers and footers
\fancyhead{} % Blank out the default header
\fancyfoot{} % Blank out the default footer
\fancyhead[C]{ \today \ $\bullet$ PyS $\bullet$ Consumo Energético de los Centros de Computación } % Custom header text
\fancyfoot[RO] {\thepage}



\begin{document}

%----------------------------------------------------------------------------------------
%	TITLE SECTION
%----------------------------------------------------------------------------------------
	\begin{titlepage}
      \centering
          %\includegraphics[width=0.15\textwidth]{example-image-1x1}\par\vspace{1cm}
          {\scshape\LARGE Universidad de Valladolid \par}
          \vspace{1cm}
          {\scshape\Large Profesión y Sociedad.\par}
          \vspace{1.5cm}
          {\huge\bfseries Consumo Energético de los Centros de Computación \par}
          \vspace{2cm}
          {\large
          \textsc{Amigo Alonso, Alberto}\\[2mm] % Your name
          \textsc{Delgado Álvarez, Sergio}\\[2mm] % Your name
          \textsc{García Prado, Sergio}\\[2mm] % Your name
          \textsc{Iglesias Cortijo, David}\\[2mm] % Your name

          \vspace{-5mm}
          }

          \vfill
		% Bottom of the page
		{\large \today\par}
	\end{titlepage}

%----------------------------------------------------------------------------------------
%	TABLE OF CONTENTS
%----------------------------------------------------------------------------------------

	\clearpage
	\tableofcontents

%----------------------------------------------------------------------------------------
%	TEXT
%----------------------------------------------------------------------------------------

	\clearpage
  \section{Introducción}
	\label{sec:introducion}
    \paragraph{}


  \section{Análisis sobre el consumo}
  \label{sec:analisis}
  	\paragraph{}
		Como es bien conocido, el gran problema de los actuales centros de datos y de supercomputación, que viene siendo arrastrado desde el pasado, es la gran cantidad de energía que necesitan para su funcionamiento. Es por ello, que el primer paso en este estudio será analizar de dónde procede esta necesidad energética y cuantificar el consumo medio de un centro de computación.

    \subsection{Dentro de un centro de computación}
			\paragraph{}

   \subsection{Consumo por elemento}
			\paragraph{}

  \section{Impacto medioambiental}
	\label{sec:impacto}

    \paragraph{}
		Los centros de computación provocan una gran alteración del medio ambiente, ya que requieren mucha energía para poder funcionar y además desprenden grandes cantidades de calor. Por tanto el impacto medioambiental es grande. A continuación se tratarán las formas en las que los centros de computación afectan gravemente al medio ambiente.

    \subsection{Gasto energético}
	  	\paragraph{}

	  \subsection{Impacto hidráulico}
	  	\paragraph{}

	  \subsection{Emisiones de dióxido carbono}
	  	\paragraph{}

	  \subsection{Impacto sobre el suelo terrestre}
	  	\paragraph{}

	  \subsection{Fauna y flora del entorno}
	  	\paragraph{}



  \section{Estrategias de optimización}
	\label{sec:estrategias}

  	\paragraph{}
		Como ya se ha visto en la sección \ref{sec:analisis}, los grandes centros de computación producen un elevado consumo energético, lo que repercute negativamente en la productividad de los mismos, y por lo tanto en los beneficios económicos. Además, tal y como vimos en la sección \ref{sec:impacto}, el entorno medioambiental en la zona donde estos se localizan puede verse afectado negativamente.


		\paragraph{}
		Debido a estos factores, las organizaciones encargadas de gestionar este tipo de centros, cada vez más, dedican un alto grado de esfuerzo para tratar de reducir su consumo energético. Existen numerosos documentos emitidos por distintas entidades de prestigio que tratan de proponer un con junto de estrategías o puntos de revisión del sistema para tratar de reducir su consumo energético.


		\paragraph{}
		En este documento vamos a dividir estas técnicas según el objetivo al que van dirigidas:


		\subsection{Refrigeración}

			\paragraph{}


		\subsection{Estado de Espera}

			\paragraph{}


		\subsection{Otras Estrategias}

			\paragraph{}

		\subsection{Optimización basada en Inteligencia Artificial}

			\paragraph{}


  \section{Beneficios}
	\label{sec:beneficios}

  	\paragraph{}

  \section{Conclusiones}
	\label{sec:conclusiones}

  	\paragraph{}
%----------------------------------------------------------------------------------------
%	Bibliographic references
%----------------------------------------------------------------------------------------
	\clearpage
  \begin{thebibliography}{9}

    \bibitem{energy-star:guide}
		Enegy Star: 12 Ways to Save Energy in Data Centers and Server Rooms. \newline \url{https://www.energystar.gov/products/low_carbon_it_campaign/12_ways_save_energy_data_center}

    \bibitem{intel:guide}
    Intel: Reducing Data Center Energy Consumption. \newline
 		\url{https://www.irif.fr/~yunes/divers/papers/green/CERN_r04.pdf}

    \bibitem{google:ia}
    Google Blog: Better data centers through machine learning. \newline
		\url{https://googleblog.blogspot.com.es/2014/05/better-data-centers-through-machine.html}

    \bibitem{google:deep-learning}
    Deepmind: Deepmind AI reduces Google Data Centre Cooling Bill by 40\%. \newline
		\url{https://deepmind.com/blog/deepmind-ai-reduces-google-data-centre-cooling-bill-40/}

    \bibitem{nrdc:guide}
    NRDC: Data Center Eficiency Assessment. \newline
 		\url{https://www.nrdc.org/sites/default/files/data-center-efficiency-assessment-IP.pdf}

		\bibitem{buildings:guide}
    Buildings: 10 Ways to Save Energy in Your Data Center. \newline \url{http://www.buildings.com/article-details/articleid/6000/title/10-ways-to-save-energy-in-your-data-center}

		\bibitem{forbes:save_energy}
		Forbes: How To Save Energy In The Data Center With Colocation And Hybrid IT. \newline \url{http://www.forbes.com/sites/centurylink/2015/12/22/how-to-save-energy-in-the-data-center-with-colocation-and-hybrid-it}

		\bibitem{energy:guide}
		Enegy: Best Practices Guide for Energy-E cient Data Center Design. \newline
		\url{https://energy.gov/sites/prod/files/2013/10/f3/eedatacenterbestpractices.pdf}

		\bibitem{ibm:guide}
		IBM: Creating a green data center to help reduce energy costs and gain a competitive advantage. \newline
		\url{https://www-935.ibm.com/services/multimedia/GTW03020USEN_186553.pdf}

		\bibitem{colocationamerica:save_energy}
		Colocation America: How Data Centers are Saving Energy. \newline
		\url{https://www.colocationamerica.com/blog/how-data-centers-save-energy}

		\bibitem{elastictree:save_energy}
		ElasticTree: Saving Energy in Data Center Networks. \newline
		\url{http://static.usenix.org/event/nsdi10/tech/full_papers/heller.pdf}

		\bibitem{nolimits:pue}
		No Limits Softwre: Data Center Energy Efficiency – Looking Beyond PUE. \newline
		\url{http://www.nolimitssoftware.com/docs/DataCenterEnergyEfficiency_LookingBeyond.pdf}

		\bibitem{wikipedia:environmental_control}
		Wikipedia: Data Center Environmental Control \newline
		\url{https://en.wikipedia.org/wiki/Data_center_environmental_control}

		\bibitem{google:case_study}
		Google: Google’s Green Data Centers: Network POP Case Study.\newline
		\url{http://static.googleusercontent.com/external_content/untrusted_dlcp/www.google.com/en/us/corporate/datacenter/dc-best-practices-google.pdf}

		\bibitem{sciencedirect:case_study}
		ScienceDirect: Data Center Energy and Cost Saving Evaluation \newline
		\url{http://ac.els-cdn.com/S1876610215009467/1-s2.0-S1876610215009467-main.pdf?_tid=d1bd2bb4-cf33-11e6-94e0-00000aab0f01&acdnat=1483173395_ac49c1e563caaf0fe7fd08f0924993f6}

		\bibitem{cisco:guide}
		Cisco: Data Center Power and Cooling \newline
		\url{http://www.cisco.com/c/en/us/solutions/collateral/data-center-virtualization/unified-computing/white_paper_c11-680202.pdf}

	\end{thebibliography}

\end{document}
